%%\iftrue
%\iffalse
%
%\documentclass[12pt, a4paper, oneside]{book}
%% todo: fix appearances
%\usepackage[backend=biber]{biblatex}
%\addbibresource{references.bib}
%
%\usepackage{comment}                            % having comment sections \begin{comment} \end{comment}
%\usepackage[utf8]{inputenc}						% charactere interpretation
%\usepackage{amsmath}							% math package
%\usepackage{amsfonts}							% font package for math symbols
%\usepackage{amssymb}							% symbols package - definition of math symbols
%\usepackage{listings}							% package for code representation
%\usepackage{csquotes}       % Quotation support
%\usepackage{graphicx}							% for inclusion of image
%\setlength {\marginparwidth }{2cm}
%\usepackage{todonotes}
%\usepackage{booktabs}       % Better tables
%\usepackage{caption}        % Better captions
%\usepackage{subfig}								% to arrange figures next to each other
%\usepackage{float}								% text style surrounding images
%\usepackage{threeparttable}
%\usepackage{tikz}								% used to place logos on title page
%% \usepackage{gensymb}							% for special characters such as °
%\usepackage{titlesec}
%\usepackage{multirow}
%\usepackage{siunitx}
%\usepackage{tabularx}
%\usepackage{tikzscale}
%% Format chapter titles without "Chapter X" prefix
%\titleformat{\chapter}[hang]
%{\normalfont\LARGE\bfseries}  % Style: Large bold text
%{\thechapter}                 % Number format: Just the number
%{1em}                         % Space between number and title
%{}                            % Code before the title (empty)
%\usepackage{hyperref}
%\hypersetup{hidelinks}
%\usepackage[acronym]{glossaries}         				% package for glossary
%\makenoidxglossaries
%\newacronym{HSLU}{HSLU}{Lucerne University of Applied Sciences and Arts}
\newacronym[see={[Glossary:]{fitzpatrick-skin-type}}]{FST}{FST}{Fitzpatrick skin type\glsadd{fitzpatrick-skin-type}}
\newacronym{ML}{ML}{Machine Learning}
\newacronym{AI}{AI}{Artificial Intelligence}
\newacronym{FPR}{FPR}{false positive rate}
\newacronym{TPR}{TPR}{true positive rate}
                                % include acronyms.txt file
%\newglossaryentry{fitzpatrick-skin-type}{
	name={Fitzpatrick skin type},
	plural={Fitzpatrick skin types},
	description={A skin classifier based on the skins' reaction to ultraviolet light, developed by dermatologist Dr. Thomas Fitzpatrick \autocite{Gottfrois2024}}
}
\newglossaryentry{JupyterNotebook}{
	name={Jupyter Notebook},
	description={Executable files, often used in ML to write Python code and add explanations in text form}
}
\newglossaryentry{gpuhub}{
	name={GPUhub},
	description={\gls{HSLU}’s server infrastructure for GPU-related computing. It provides isolated environments with JupyterLab access for developing and running \gls{ML} workflows}
}
\newglossaryentry{pediatric}{
	name={pediatric},
	description={A medical term for infants, children and adolescents \autocite{Farlex_nodate}}
}
\newglossaryentry{proxyVar}{
	name={proxy variable},
	plural={proxy variables},
	description={"one or more variables that encode the protected attribute with a substantial degree of accuracy" \autocite{Wang_2021}}
}
\newglossaryentry{teledermatology}{
	name={teledermatology},
	description={dermatological care from a distance, supported by modern technology \autocite{Pala_2020}}
}
\newglossaryentry{Fairlearn}{
	name=Fairlearn,
	description={A Python library for assessing and improving fairness in machine learning models. It supports various fairness metrics and mitigation techniques, especially for binary classification tasks \autocite{Fairlearn_nodate}}
}
\newglossaryentry{Equalized-Odds-Difference}{
	name={equalized odds difference},
	description={The absolute difference in true positive and false positive rates between subgroups, used as a group fairness metric \autocite{Fairlearn_nodate}}
}
\newglossaryentry{Equalized-Odds-Ratio}{
	name={equalized odds ratio},
	description={The ratio of true positive and false positive rates between subgroups, used as a group fairness metric \autocite{Fairlearn_nodate}}
}                                % include glossary.txt file
%\graphicspath{{figures/}}						    % set path of graphics folder
%% mentioned in header
%\newcommand{\tblWidthDescription}{\hsize=0.6\hsize\raggedright}
%\newcommand{\tblWidthContext}{\hsize=0.2\hsize}
%%improved basic functionality
%\newcommand{\bolditalic}[1]{\textbf{\textit{{#1}}}}
%%indicate citations
%% Define a flag to track whether we're inside a raw citation block
%\newif\ifrawcitationactive
%\rawcitationactivefalse % Default: Not inside a raw citation block
%% Define color commands with conditional checking
%\newcommand{\rawcitationstart}{
%	\color{purple}\rawcitationactivetrue
%}
%\newcommand{\rawcitationend}{
%	\color{black}\rawcitationactivefalse
%}
%\newcommand{\rawcitationusedstart}{\color{violet}}
%\newcommand{\rawcitationusedend}{%
%	\ifrawcitationactive
%	\color{purple}  % If inside rawcitation, reset to purple
%	\else
%	\color{black}  % Otherwise, reset to black
%	\fi
%}
%\begin{document}
%\fi
	
	\begin{refsection}
		
		
		\section{initial sources}
		\todo{check that all stuff above matches the stuff below}
		
					\rawcitationstart
		\subsection{Bias Introduction}
		\begin{itemize}
			\item \textbf{Assessment Tools} An interesting direction that researchers have taken is introducing tools that can assess the amount of fairness in a tool or system. For example, Aequitas [136] is a toolkit that lets users to test models with regards to several bias and fairness metrics for different population subgroups. Aequitas produces reports from the obtained data that helps data scientists, \gls{ML} researchers, and policymakers to make conscious decisions and avoid harm and damage toward certain populations. \gls{AI} Fairness 360 (AIF360) is another toolkit developed by IBM in order to help moving fairness research algorithms into an industrial setting and to create a benchmark for fairness algorithms to get evaluated and an environment for fairness researchers to share their ideas [11]. These types of toolkits can be helpful for learners, researchers, and people working in the industry to move towards developing fair \gls{ML} application away from discriminatory behavior \autocite{Mehrabi_2021}.
		\end{itemize}	
		
		\begin{itemize}
			\item Most \gls{AI} systems and algorithms are data driven and require data upon which to be trained. Thus, data is tightly coupled to the functionality of these algorithms and systems. In the cases where the underlying training data contains biases, the algorithms trained on them will learn these biases and reflect them into their predictions. As a result, existing biases in data can affect the algorithms using the data, producing biased outcomes. Algorithms can even amplify and perpetuate existing biases in the data.\autocite{Mehrabi_2021}.
			\item In addition, algorithms themselves can display biased behavior due to certain design choices, even if the data itself is not biased. The outcomes of these biased algorithms can then be fed into real-world systems and affect users’ decisions, which will result in more biased data for training future algorithms.\autocite{Mehrabi_2021}.
			
			\item Bias can exist in many shapes and forms, some of which can lead to unfairness in different downstream learning tasks. In \autocite{M144_Suresh_2021}, authors talk about sources of bias in \gls{ML} with their categorizations and descriptions in order to motivate future solutions to each of the sources of bias introduced in the paper. In \autocite{M120_Olteanu_2019}, the authors prepare a complete list of different types of biases with their corresponding definitions that exist in different cycles from data origins to its collection and its processing.\autocite{Mehrabi_2021}.
		\end{itemize}	
		
		\begin{itemize}
			\item The list of biases that can occur in any research is considerably long, and certainly not all of them can be avoided. \autocite{Chakraborty_2024}
		\end{itemize}
		\rawcitationend
		
		\rawcitationstart
		\subsection{Biases Extensive Sources}
		
		\paragraph{Data Biases}
		Data biases (data to algorithm (biases in data which might have an impact on biased algorithmic outcomes \autocite{Mehrabi_2021}))	
		\begin{itemize}
			\item 
		\end{itemize}	
		
		
		\paragraph{Algorithmic Biases}
		
		Algorithmic biases (Algorithm to user (A modulates U behaviour, biases in algorithm might lead to introduce biases in user behaviour and affect it as a consequence)) \autocite{Mehrabi_2021}
		
		\paragraph{User Biases}
		
		User to Data (user-generated data, inherent biases in users could be reflected in the data they generate; biases in last section might introduce further bias in this process) \autocite{Mehrabi_2021}
		
		\paragraph{Dermatology Biases}
		\begin{itemize}
			\item Equity. \gls{AI} has the potential to worsen health-care disparities, as recognized by the popular media (Khullar, 2019), particularly in dermatology (Adamson and Smith, 2018). The first concern is adequate representation of underserved populations in training data. Existing DL models have been trained on mainly European or East Asian populations, and the relative lack of training on darker skin pigmentation may limit overall diagnostic accuracy. This possibility is demonstrated by the increased error rates in commercial systems, trained on predominantly white datasets, for facial analysis in identifying black individuals (Buolamwini and Gebru, 2018). Second, \gls{AI} may entrench existing social and economic biases and perpetuate inadvertent discriminatory practices, for example, in recommending less follow-up for black patients than for whites, when health costs are used as a proxy for health needs (Obermeyer et al., 2019). Third, disproportionate adoption by different groups may exacerbate existing inequities. Access to and use of technology differs based on sociodemographics (Tsetsi and Rains, 2017), and more techsavvy users may be more likely to embrace \gls{AI} for skin screening (Tong and Sopory, 2019). The issue of equity in \gls{AI} diagnosis needs to be carefully addressed to avoid inadvertent exacerbation of health-care disparities. \autocite{Young_2020} - dermatology
			\item Model generalizability. Generalizability is a major concern for \gls{AI} models; studies of computer-assisted diagnosis of melanoma report lower sensitivity for melanoma on independent test sets than on nonindependent test sets (Dick et al., 2019). It is difficult to study generalizability because published DL models are not publicly available, making it impossible to compare performance, unless each study uses a standardized benchmark database, such as the Melanoma Classification Benchmark (Brinker et al., 2019d). Han et al. (2018a) reported excellent metrics of performance and made their model available for image submission; however, the model prediction was not robust when images from an outside clinic were submitted, image magnification or contrast was altered, or images were rotated (NavarreteDechent et al., 2018). On ImageNet, a nonmedical dataset of 1,000 object categories, training on a dataset of 300 ... \autocite{Young_2020}
		\end{itemize}
		\rawcitationend
		
		\rawcitationusedstart
		\begin{itemize}
			\item two potential sources of unfairness in \gls{ML} outcomes - those that arise from biases in the data and those that arise from the algorithms ... we observe that biased algorithmic outcomes might impact user experience, thus generating a feedback loop between data, algorithms and users that can perpetuate and even amplify existing sources of bias \autocite{Mehrabi_2021}.
			\item The loop capturing this feedback between biases in data, algorithms, and user interaction is illustrated in Figure 1. We use this loop to categorize definitions of bias in the section below \autocite{Mehrabi_2021}
		\end{itemize}
		\rawcitationusedend
		
		
		\subsection{Bias Types}
		
		\todo{Feedback Astrid: zu viele Unterkapitel -> anders strukturieren}
		\subsubsection{Data Biases}
		
		\paragraph{Sampling Biases}
		When gathering data, it's usually not possible to gather the data of a whole population. Instead, the data is gathered by sampling. A sample is a subgroup of individuals from the population. To get unbiased results, this sampling process should represent the true population, with a low sampling error \autocites{HP_2022}. This is often achieved with randomized samples. With non-random sampling processes, sampling bias arises. The consequence is, that the insights of one sampled population may not generalize with insights on another sampled popluation \autocite{Mehrabi_2021}.
		
		Those biases can be introduced with a flawed sampling process:
		\begin{itemize}
			\item \textbf{Sampling bias}, due to nonrandom sampling of subgroups, leading to poor generalization \autocite{Mehrabi_2021}
			\item \textbf{Selection bias}, working only on specific subset of the population which is not representative \autocites{Mester_2022}{Chakraborty_2024}
			\item \textbf{Systematic selection bias}, chosen samples differ dramatically from the representative populations; e.g. in dermatology, when only the most severe patient data gets included \autocite{Chakraborty_2024, c5,c6,c33}
			\item \textbf{Ascertainment bias}, tendency to exclude segments from the population due to e.g. cultural differences, such as which patient segment goes to government clinics vs. private clinics (usually influenced by socioeconomic status) \autocite{Chakraborty_2024, c5}
			\item \textbf{Availability bias}, focus on widely available data instead of most representative data \autocites{Chakraborty_2024, c9, c10}{}
			\item \textbf{Survivorship bias}, focus only on pre-selected data, ignoring the initial data-points which got filtered out \autocite{Mester_2022}.
		\end{itemize}
		
		
		\subparagraph{Potential Biases in PASSION}
		PASSION tries to reduce sampling bias in dermatology against high pigmented skin.
		PASSION might introduce (systematic) selection bias or Ascertainment bias, if in the dermatology centers only sickest / more severe patients are seen as indicated by \textcite{Chakraborty_2024}
		PASSION inherits availability bias as it is using \gls{FST} scale.
		Survivorship bias could be relevant for PASSION, if dermatology diseases could be lethal. Further, all patients which are not able to go to one of the dermatology centers which were used in PASSION could be considered to left out by survivorship bias.
		
		\rawcitationstart
		used
		\begin{itemize}		
			\rawcitationusedstart
			\item Sampling Bias. Sampling bias is similar to representation bias, and it arises due to nonrandom sampling of subgroups. As a consequence of sampling bias, the trends estimated for one population may not generalize to data collected from a new population. \autocite{Mehrabi_2021}. This is what the PASSION dataset tries to improve
			
			\item Selection bias - wrong sampling method, working on a specific subset of audience; usually by working only with data that is easy to access \autocites{Mester_2022}{Mester_2017} - statistical bias
			\item Selection bias: Since it is not possible to work with large populations, for most dermatological studies, samples are chosen that are said to be representative of the original population. 
			In selection bias, the selected subgroups are not representative of their original population.
			A variation of this is systematic selection bias, where samples chosen differ dramatically from their representative populations.
			Our experience suggests, such selection bias occurs more commonly in studies conducted in regional referral centers where only the sickest or more severe patients are usually seen.
			For example, a study compared the efficacy of thalidomide vs. prednisolone in hospitalised patients of erythema nodosum leprosum. It derived that thalidomide was more efficacious than steroids in erythema nodosum leprosum. Such findings cannot be generalized to all erythema nodosum leprosum since patients admitted to a regional referral center will likely have more severe disease.5,6,33 \autocite{Chakraborty_2024}
			\item  Availability bias: More emphasis is placed on widely available data than scantily available data. A classic example is the use of antihistamines in pregnancy dermatoses, where nearly all standard books recommend first-generation antihistamine chlorpheniramine because more data is available.9 10. \autocite{Chakraborty_2024} - dermatology
			
			\item Survivorship bias \autocites{Mester_2022}{Mester_2017} - statistical bias
			
			
			\item Ascertainment Bias: This bias is commonly encountered in venereology practice. It is defined as a bias due to the tendency of some segments of the target population to get excluded due to cultural and other differences. For example, in most venereology clinics in government setups, studies show that venereal diseases are commoner in lower socioeconomic status. One reason might be that the higher socioeconomic status people tend to go to private practitioners and thereby get excluded from government-run clinics.9,10 Allocation concealment and blinding are good ways to avoid this. 5. \autocite{Chakraborty_2024} - healthcare
			\rawcitationusedend
		\end{itemize}
		
		even more extensive
		\begin{itemize}
			\item Selection bias is again divided into two types endogenous selection bias and exogenous selection bias. The best example of endogenous selection bias in dermatology is the inclusion of non-response. If a trial tests the efficacy of a particular biologic in psoriasis, the response is usually collected from trial participants via postal services. Certain participants will not respond, although they might have substantially improved. Their exclusion will result in significant differences in efficacy evaluation.33
			Exogenous selection bias results when both treatment and outcome result from dependency on an external variable that is not controlled. For example, if sunlight exposure is not controlled, it will influence both the intervention and control groups since psoriasis is a photosensitive (and photoexcerbated) dermatosis. \autocite{Chakraborty_2024} - dermatology
			
			
			\item survivorship bias - World War II planes \autocite{Silfwer_2017} - https://doctorspin.org/media-psychology/psychology/survivorship-bias/
		\end{itemize}
		
		\rawcitationend
		
		\paragraph{Representation Biases}
		\todo{still describe this category} 
		
		Those biases can be introduced :
		\begin{itemize}
			\item \textbf{Representation bias}, non-representative sample lead to missing subgroups or other representation anomalies, which can be harmful to downstream applications. Popular \gls{ML} datasets suffer from representation bias \autocites{Mehrabi_2021}{M142_Shankar_2017}
			\item \textbf{Population Bias}. Population bias arise when statistics, demographics and characteristics in the sample differ from the target population \autocite{M120_Olteanu_2019}. The data it creates is non-representative for the target population \autocite{Mehrabi_2021}.
			\item \textbf{Aggregation bias} occurs, when "false conclusions are drawn about individuals from observing the entire population". It doesn't matter, whether the subgroups are represented equally in the training set, any generalized assumptions can result in aggregation bias \autocite{Mehrabi_2021}. In medicine, diseases can present themselves differently across sexes and ethnicities \autocite{M144_Suresh_2021}. Therefore, diagnostic models need to incorporate those differences to mitigate aggregation bias \autocite{Mehrabi_2021}.
			\item \textbf{Simpson's Paradox} is a type of aggregation bias, which arises in heterogeneous data analysis. Observed associations disappear or reverses in the subgroup data \autocite{Mehrabi_2021}.
		\end{itemize}
		
		
		\subparagraph{Potential Biases in PASSION}
		PASSION tries to mitigate representation bias, by including more FST skin types - however, it could introduce other representation biases
		Aggregation bias and Simpson's Paradox could potentially be an issue when the analyzed skin diseases present themselves differently in patients based on their genetics
		
		
		\rawcitationstart
		used
		\begin{itemize}		
			\rawcitationusedstart
			\item Representation Bias. Representation bias arises from how we sample from a population during data collection process \autocite{M144_Suresh_2021}. Non-representative samples lack the diversity of the population, with missing subgroups and other anomalies \autocite{Mehrabi_2021}.
			\item Popular machine-learning datasets that serve as a base for most of the developed algorithms and tools can also be biased—which can be harmful to the downstream applications that are based on these datasets. ... In \autocite{M142_Shankar_2017}, researchers showed that these datasets suffer from representation bias and advocate for the need to incorporate geographic diversity and inclusion while creating such datasets. \autocite{Mehrabi_2021}
			
			\item Population Bias. Population bias arises when statistics, demographics, representatives, and user characteristics are different in the user population of the platform from the original target population \autocite{M120_Olteanu_2019}. Population bias creates non-representative data. ... More such examples and statistics related to social media use among young adults according to sex, race, ethnicity, and parental educational background can be found in \autocite{M64_Hargittai_2007}. \autocite{Mehrabi_2021}
			
			\item Aggregation Bias. Aggregation bias (or ecological fallacy) arises when false conclusions are drawn about individuals from observing the entire population. An example of this type of bias can be seen in clinical aid tools. Consider diabetes patients who have apparent morbidity differences across ethnicities and sexes. Specifically, HbA1c levels, that are widely used to diagnose and monitor diabetes, differ in complex ways across sexes and ethnicities. Therefore, a model that ignores individual differences will likely not be well-suited for all ethnic and sex groups in the population \autocite{M144_Suresh_2021}. This is true even when they are represented equally in the training data. Any general assumptions about subgroups within the population can result in aggregation bias. \autocite{Mehrabi_2021}. --> could also be important for dermatology issues!!!
			\begin{itemize}
				\item Simpson’s Paradox. Simpson’s paradox is a type of aggregation bias that arises in the analysis of heterogeneous data [18]. The paradox arises when an association observed in aggregated data disappears or reverses when the same data is disaggregated into its underlying subgroups (Fig. 2(a)). ... After analyzing graduate school admissions data, it seemed like there was bias toward women, a smaller fraction of whom were being admitted to graduate programs compared to their male counterparts. However, when admissions data was separated and analyzed over the departments, women applicants had equality and in some cases even a small advantage over men. The paradox happened as women tended to apply to departments with lower admission rates for both sexes. Simpson’s paradox has been observed in a variety of domains, including biology [37], psychology [81], astronomy [109], and computational social science [91].\autocite{Mehrabi_2021}.
			\end{itemize}
		\end{itemize}
		\rawcitationusedend
		\rawcitationend
		
		\paragraph{Measurement Biases}
		How features are chosen, used and measured can lead to biases \autocites{Mehrabi_2021}{M144_Suresh_2021}.
		
		Examples for such biases are:
		\begin{itemize}
			\item \textbf{Measurement bias} in general, e.g. using mismeasured \glspl{proxyVar} lead to misinterpretations of the outcome \autocite{Mehrabi_2021}
			
			\item \textbf{Observer bias} is a subconscious bias which can occur in different forms. Either, researchers projects their own expectations on the research and influence the testers accordingly \autocite{Mester_2022}. In other cases, different observes report the same observation differently \autocite{Chakraborty_2024, c29, c26}
			
			\item \textbf{Annotator bias} is a special form of observer bias. The labeling process of human annotators can be influenced by lots of factors (e.g. personal background, social context) and even minor design choices (e.g. scale order, image context). This can introduce inconsistencies when labeling the data \autocite{Montoya_2025}
			
			\item \textbf{Recall bias}. This bias occurs when queried individuals do not remember things correctly, due to humans selective memory. This can cause misinterpretation, for example when analyzing causes and effects of behaviour on certain diseases in medicine \autocites{Mester_2022}{Chakraborty_2024, c3-6, c2}.
		\end{itemize}
		
		\subparagraph{Potential Biases in PASSION}
		Measurement Bias (proxy var) - Country of Origin in PASSION depending on the interpretation - should not be used for ethnicity, as this is not linked directly to the genes, see example https://medium.com/bcggamma/practice-ai-responsibly-with-proxy-variable-detection-42c2156ad986
		
		Annotator bias regarding skin tone labeling has been investigated in \autocite{Montoya_2025}. PASSION should evaluate its process.
		
		
		\rawcitationstart
		used
		\begin{itemize}		
			\rawcitationusedstart
			\item Measurement Bias. Measurement, or reporting, bias arises from how we choose, utilize, and measure particular features \autocite{M144_Suresh_2021} (e.g. mismeasured proxy variables) \autocite{Mehrabi_2021}. (= e.g. someone who lives at that postal code probably has this ethnicity ); --> could that be an issue with the country of origin feature?
			
			\item This study found that while using skin tone instead of race for fairness evaluations in computer vision seems objective, the annotation process remains biased by human annotators. Untested scales, unclear procedures, and a lack of awareness about annotator backgrounds and social context significantly influence skin tone labeling. This study exposes how even minor design choices in the annotation process, like scale order (dark to light instead of light to dark) or image context (face or no face, skin lesion presence), can sway agreement and introduce uncertainty in skin tone assessments. ... The researchers emphasize the need for greater transparency, standardized procedures, and careful consideration of annotator biases to mitigate these challenges and ensure fairer and more robust evaluations in computer vision. \autocite{Montoya_2025} - demographic dermatology bias
			
			\item Observer bias - projecting expectations onto the research \autocites{Mester_2022}{Mester_2017} - statistical bias
			\item  Observer bias: When different observers view the same observation, they report it differently e.g., different observers may give differing descriptions about subtle features in the histopathology report of a skin biopsy.29 26. \autocite{Chakraborty_2024} - dermatology
			
			\item Recall bias - respondent doesn't remember things correctly; Recall bias is another common error of interview/survey situations. It happens when the respondent doesn’t remember things correctly. It’s not about bad or good memory – humans have selective memory by default. After a few years (or even a few days), certain things stay and others fade. It’s normal, but it makes research much more difficult. \todo{keep an eye on this when recalling evidences!!}
			\autocites{Mester_2022}{Mester_2017} - statistical bias
			\item Memory or recall bias: This is a type of bias where sufferers of a disease, often termed cases, have a greater tendency to recall a particular habit than non-sufferers, viz controls. This results in an uneven distribution of risk factors between the cases and controls. An example of this would be a case-control study to evaluate the association between dental amalgam use and the development of oral lichen planus. Those with lichen planus are more likely to recall a history of dental amalgam use than those who do not have the disease. This difference in recall between a diseased cohort and control has resulted in difficulties in assessing the association between diet and many dermatological diseases – like milk and chocolate consumption and acne, fatty meals and psoriasis, sugary meals and psoriasis, agricultural exposure to insecticides and pemphigus and so on.3–6 2. \autocite{Chakraborty_2024} - dermatology
			
			\rawcitationusedend
		\end{itemize}
		\rawcitationend
		
		\paragraph{Research Biases}
		\todo{consider to move at beginning / out of data biases}
		Researchers and their processes can also be biased in multiple ways:
		\begin{itemize}
			\item \textbf{Funding / Sponsorship bias}, when a study is deliberately supporting those findings, which the sponsor expects \autocites{Chakraborty_2024, c22}{Mester_2017}
			
			\item \textbf{Data dredging bias}. The statistical methods and model are chosen to provide a certain p-value, to improve the probability of the research hypothesis being true. \todo{consider to move this to an own reporting section} \autocite{Chakraborty_2024}
			
			\item \textbf{Hypothetical bias}. Hypothetical questions lead to responses that do not reflect, what interviewees would do in real life. \autocite{Chakraborty_2024, c31, c28} \todo{isn't this a user bias instead?}
		\end{itemize}
		
		\subparagraph{Potential Biases in PASSION}
		Since the PASSION dataset is already published, the research biases might already be introduced. It is not feasible during the duration of this thesis to make an evaluation on those biases. Instead, I would recommend the PASSION team and researcher in general, to check the list above carefully and take measures against them. Maybe, an external evaluation could help to detect and prevent those biases even better.
		
		
		
		\rawcitationstart
		used
		\begin{itemize}		
			\rawcitationusedstart
			\item Funding bias \autocites{Mester_2022}{Mester_2017} - statistical bias
			\item  Industry sponsorship bias: This has now been reclassified as conflict-of-interest bias. In short, the study deliberately supports the findings expected from it by its sponsors. 22.\autocite{Chakraborty_2024} - dermatology
			
			Reporting biases
			\item  Data dredging bias: It is an entirely avoidable bias. This is subdivided into two types – Fishing type and “P-value hacking” type. It involves using multiple statistical methods to get the desired p-value and selecting the statistical model that gives the p-value the author wants. This is “lamentably common” in dermatological research.16 To detect data dredging bias, always perform a “p-curve analysis” while performing a meta-analysis.17,18 Much emphasis is nowadays given to the confidence interval instead of the p-value, which gives an approximate idea of the range in which one can be 95\% (or 90\%, depending on the confidence interval chosen) sure that the result is correct. The confidence interval remains unaffected by p-value dredging. This subject has been reviewed in depth in recent works.18,19 15.\autocite{Chakraborty_2024}
			
			\item Hypothetical bias: Many dermatological researches (and some life quality questionnaires like vitiQoL) use hypothetical questions – like “What would you do when some stranger asks you about your lesion?”. The responses to these questions by the study participants often do not tally with what they would do in real life. This is called hypothetical bias and is avoided by adopting the ex-ante approach.31 28. \autocite{Chakraborty_2024} - dermatology
			\rawcitationusedend
		\end{itemize}
		\rawcitationend
		
		\paragraph{Feature Representation Biases}
		
		Some of those biases are:
		\begin{itemize}
			\item \textbf{Omitted Variable Bias} arises when variables are not included in the model, which leads to situations for which the model is not ready for \autocites{Mehrabi_2021}{Mester_2022}\autocites{M38_Clarke_2005}{M131_Riegg_2008}\autocite{M114_Mustard_2003}.
			\item \textbf{Collider Bias} Two variables can influence a common third variable, the collider variable. When sampling is restricted by this collider variable, it could lead to a distortion  \autocite{Chakraborty_2024, c4,c8,c9}.
		\end{itemize}
		
		
		\subparagraph{Potential Biases in PASSION}
		The ethnicity is omitted in the PASSION dataset which could lead to issues
		See the medical section for more specific collider bias, maybe there could be others
		
		\rawcitationstart
		used
		\begin{itemize}
			\rawcitationusedstart
			\item Omitted Variable Bias. Omitted variable bias4 occurs when one or more important variables are left out of the model \autocites{M38_Clarke_2005}{M131_Riegg_2008}\autocite{M114_Mustard_2003}. Something that the model was not ready for\autocite{Mehrabi_2021}. did not take into account \autocite{Mehrabi_2021}
			\item Omitted variable bias \autocites{Mester_2022}{Mester_2017} - statistical bias
			\item Collider Bias: This is an under-appreciated bias, and often confused with a confounder. This is especially seen in observational studies where it is defined as a distortion produced by the restriction of sampling by a collider variable. A collider variable is defined as one that has an independent effect on the outcome studied apart from the studied variable. In simpler terms, collider bias occurs when exposure and development influence a common third variable. That variable or collider is controlled by study design or in the analysis. An example is the observation that psoriasis patients tend to have more depression and anxiety disorders. Since severe psoriasis patients tend to get hospitalised and also get screened for mental health issues, a spurious association between them could have been obtained due to collider bias. The two variables viz psoriasis and depression converged, i.e., collided, into a single outcome – hospitalization.8,9 4. \autocite{Chakraborty_2024} - dermatology
			\rawcitationusedend
		\end{itemize}
		\rawcitationend
		
		
		\paragraph{Imaging Biases}
		Dealing with images can lead to a whole other set of challenges, which can lead to biases. The challenges are for example technical variations in hardware and software but also differences in how images are gathered or what is in it \autocite{Young_2020}.
		
		Those biases can be introduced :
		\begin{itemize}
			\item \textbf{Image Quality Bias}. The quality of an image (zoom level, focus, lightning) could be associated with the classification \autocite{Young_2020}
			\item \textbf{Visual Artifact Bias}. Other artifacts, such as presence of hair or surgical ink markings on dermatology images, can decrease classification performance \autocite{Winkler et al., 2019 & Bisla et al., 2019 (from Young_2020)}
			\item \textbf{Field of View Bias}. What view is captured in the image can interfere with prediction quality  what is it, consequence \autocite{Mishra et al., 2019 from Young_2020}
		\end{itemize}
		
		
		\subparagraph{Potential Biases in PASSION}
		The PASSION model could learn to associate unrelated visual effects, hair, body parts or image quality with a disease.
		
		
		\rawcitationstart
		used
		\begin{itemize}		
			\rawcitationusedstart
			\item Image quality. Several barriers to \gls{AI} implementation in the clinic need to be overcome with regards to imaging (Figure 1). These include technical variations (e.g., camera hardware and software) and differences in image acquisition and quality (e.g., zoom level, focus, lighting, and presence of hair). For example, the presence of surgical ink markings is associated with decreased specificity (Winkler et al., 2019), field of view can significantly affect prediction quality (Mishra et al., 2019), and classification performance improves when hair and rulers are removed (Bisla et al., 2019). We have developed a method to measure how model predictions might be biased by the presence of a visual artifact (e.g., ink) and proposed methods to reduce such biases (Pfau et al., 2019). Poor quality images are often excluded from studies, but the problem of what makes an image adequate is not well studied. Ideally, models need to be able to express a level of confidence in a prediction as a function of image quality and appropriately direct a user to retake photos if needed. \autocite{Young_2020} - dermatology
			\rawcitationusedend
		\end{itemize}
		\rawcitationend
		
		\paragraph{Medical Biases}
		In \gls{ML} for health care, there are special medical versions of the mentioned biases as well as completely new biases. They require special attention, since they directly influence the diagnosis or treatment of a disease.
		
		Those biases can be introduced:
		\begin{itemize}
			\item \textbf{Berkesonian bias} occurs in hospital-based studies when two variables influence hospital or clinical attendance independently. This can lead to a distorted estimation of the relationship between those variables because the study population of hospitalized patients is not representative of the whole population \autocite{Chakraborty_2024, c3, c7}
			\item \textbf{Informed presence bias}, the probability to get screened for other diseases is higher for people who seek medical care. Like Berkesonian bias, this can lead to misleading interpretations of relationships between two diseases \autocite{Chakraborty_2024, c27, c23}
			\item \textbf{Diagnostic access bias}, depending on the geographical location, individuals have better access to medical care. Therefore, their disease prevalence could appear to be higher and diseases could be diagnosed earlier. \autocite{Chakraborty_2024, c19-c21}
			\item \textbf{Diagnostic reference test bias} is a \textbf{verification bias}, where not all individuals receive the same reference test for the diagnostic process, potentially leading to different diagnoses. \autocite{Chakraborty_2024, c21}
			\item \textbf{Mimicry bias}, exposures to treatment options can cause a disease which presents itself similar to the study disease, which potentially creates misleading data \autocite{Chakraborty_2024, c28, c25}
			\item \textbf{Unacceptable Disease bias}. When a disease is socially unacceptable, it can result in under-reporting of the same disease \autocite{Chakraborty_2024, c30, c27}
			\item \textbf{Healthy volunteer selection bias}, is a type of self-selection bias where the volunteers are in general healthier than the population due to more interest in health \autocite{Delgado-Rodriguez_2004}
		\end{itemize}
		
		\subparagraph{Potential Biases in PASSION}
		Berkesonian bias depending on the chosen hospitals
		Informed presence bias regarding correlation between impedigo and the other diseases
		Diagnostic access bias can somewhat be addressed by PASSION, since its dataset includes samples of later states of diseases. However, in the PASSION context itself, this bias could still be relevant.
		Diagnostic reference test bias could be inherited in the PASSION dataset, depending on how the dermatologists work.
		Mimicry bias is not relevant regarding the exposures since PASSION does not hold any exposure data. However, diseases which mimicry others could lead to issues if they are not detected.
		
		
		
		\rawcitationstart
		used
		\begin{itemize}		
			\rawcitationusedstart
			\item Berkesonian Bias: Named after Dr. Joseph Berkeson, this bias reflects the variation in rates of hospital admission or clinic attendance for different diseases. For example, if a study is conducted to know the effect of pregnancy on syphilis in an antenatal clinic, we are likely to get biased data since the two conditions, viz pregnancy and syphilis, are both likely to affect clinic attendance and all observations related to the relationship between pregnancy and syphilis.7 3. \autocite{Chakraborty_2024} - dermatology
			
			\item  Informed presence bias: Simply, a person attending a health center is more likely to get screened for other unrelated comorbidities than those not attending a health center e.g., the finding psoriasis is associated with depression has now been criticised because those having psoriasis also have a greater chance to be screened for depression since they are already attending a health center.27 23. \autocite{Chakraborty_2024} - dermatology	
			
			\item  Diagnostic Access Bias: Individuals in certain geographical localities have better access to medical care and, hence, may appear to have higher disease prevalence. For example, atopic dermatitis is believed to be commoner in the West – this could be due to better and earlier diagnostic facilities available than in India.19,20 17.\autocite{Chakraborty_2024}
			
			\item  Diagnostic reference test bias: These bias results when all individuals do not receive the same reference test. e.g., direct immunofluorescence studies may not be done for all patients with pemphigus vulgaris some patients may receive only a skin biopsy-based diagnosis. It is a subtype of verification bias. Another variation of this type of bias is partial reference bias, where only some of the study participants receive the index and the reference tests.21\autocite{Chakraborty_2024}
			
			\item  Mimicry bias: When an exposure causes a disease that resembles the study disease, mimicry bias can result. For example, certain drugs are known to cause a pityriasis rosea-like reaction, which, although looks like pityriasis rosea, differs from it.28 25.\autocite{Chakraborty_2024} - dermatology
			
			\item Unacceptable disease bias: This occurs in socially unacceptable diseases like leprosy and STDs, which result in under-reporting.30 27. \autocite{Chakraborty_2024} - dermatology
			
			\item \todo Other such studies were conducted in [\autocite{M54_Fry_2017}] which states that UK Biobank, a large and widely used genetic dataset, may not represent the sampling population. Researchers found evidence of a “healthy volunteer” selection bias. [150] has other examples of studies on existing biases in the data used in the medical domain. [157] also looks at machine-learning algorithms and data utilized in medical fields, and writes about how artificial intelligence in health care has not impacted all patients equally.\autocite{Mehrabi_2021} --> [150] also provides an ovverview over the impact of social determinants on health, such as Economic stability, neighborhood and physical environment, education, food, community and scial context, access to healthcare and quality
			\item The healthy volunteer effect is a particular case: when the participants are healthier than the general population. \autocite{Delgado-Rodriguez_2004}
			\rawcitationusedend
		\end{itemize}
		\rawcitationend
		
		\paragraph{Temporal Biases}
		Differences in populations and their behaviour over time can lead to temporal biases \autocite{M120_Olteanu_2019}.
		Certain studies require to track temporal data, to learn about their behaviour over time. Disease progression is also a factor measured over time \autocite{Mehrabi_2021}. For PASSION, temporal biases are currently irrelevant, since PASSION contains images independently of time and is not tracking the disease progression. Therefore, the listed biases in this chapter are not explained in detail, refer to the sources for further information.
		
		Examples for temporal data biases are:
		\begin{itemize}
			\item \textbf{Longitudinal Data Fallacy} \autocite{Mehrabi_2021}
			\item \textbf{Chronological bias} \autocite{Chakraborty_2024, c9, c13}
			\item \textbf{Immortal time bias} \autocite{Chakraborty_2024, c24, c20}
		\end{itemize}
		
		
		\todo{added until here}
		\subsection{Algorithmic Biases}
		When an algorithm adds biases to unbiased input data one speaks of \textbf{Algorithmic Bias} \autocite{M9_Baeza-Yates_2018}. This could occur due to algorithmic design choices like optimization functions, regularizations and statistically biased estimators \autocite{M44_Danks_2017}.
		
		\rawcitationstart
		used
		\begin{itemize}		
			\rawcitationusedstart
			\item Algorithmic Bias. Algorithmic bias is when the bias is not present in the input data and is added purely by the algorithm \autocite{M9_Baeza-Yates_2018}. The algorithmic design choices, such as use of certain optimization functions, regularizations, choices in applying regression models on the data as a whole or considering subgroups, and the general use of statistically biased estimators in algorithms \autocite{M44_Danks_2017}, can all contribute to biased algorithmic decisions that can bias the outcome of the algorithms.\autocite{Mehrabi_2021}.
			\rawcitationusedend
		\end{itemize}
		\rawcitationend
		
		\paragraph{User Algorithm Interaction Biases}
		\begin{itemize}
			\item \textbf{User Interaction Bias}. This biases can be triggered by the user interface or the user themselves. The user interface influences the user to behave in a certain way, which could introduce bias in the user behaviour. Users impose this (or their own) biased behavior through interaction on the algorithm \autocite{M9_Baeza-Yates_2018}. \textbf{Presentation bias} and \textbf{Ranking bias} are further subtypes mentioned by \textcites{M93_Lerman_2014}{Mehrabi_2021}.
			\item \textbf{Emergent Bias}. When real users interact with an algorithm, this bias arises some time after the design was completed due to changes in population. It appears more likely in user interfaces \autocite{M53_Friedman_1996}.
		\end{itemize}
		
		
		\subparagraph{Potential Biases in PASSION}
		The user interaction biases, especially the emergent bias could potentially become an issue for PASSION, when the project starts to become publicly available \gls{teledermatology}. Also, the interface design should be evaluated, so that no presentation or ranking bias gets introduced.
		
		\rawcitationstart
		used
		\begin{itemize}		
			\rawcitationusedstart
			\item Emergent Bias. Emergent bias occurs as a result of use and interaction with real users. This bias arises as a result of change in population, cultural values, or societal knowledge usually some time after the completion of design \autocite{M53_Friedman_1996}. This type of bias is more likely to be observed in user interfaces, ... This type of bias can itself be divided into more subtypes, as discussed in detail in \autocite{M53_Friedman_1996}. \autocite{Mehrabi_2021}. probably less relevant at the first stage
			
			\item User Interaction Bias. User Interaction bias is a type of bias that can not only be observant on the Web but also get triggered from two sources—the user interface and through the user itself by imposing his/her self-selected biased behavior and interaction \autocite{M9_Baeza-Yates_2018}. This type of bias can be influenced by other types and subtypes, such as presentation and ranking biases. \autocite{Mehrabi_2021}. -- more relevant for later, when the application would become bigger
			\rawcitationusedend
			\begin{itemize}
				\item Presentation Bias. Presentation bias is a result of how information is presented \autocite{M9_Baeza-Yates_2018} (can only click on content they see, could be the case that user does not see all info on web) \autocite{Mehrabi_2021}.
				\item Ranking Bias. The idea that top-ranked results are the most relevant and important will result in attraction of more clicks than others. This bias affects search engines \autocite{M9_Baeza-Yates_2018} and crowdsourcing applications \autocite{M93_Lerman_2014}.\autocite{Mehrabi_2021}.
			\end{itemize}
		\end{itemize}
		\rawcitationend
		
		\paragraph{External Influence Biases}
		
		Those biases can be introduced :
		\begin{itemize}
			\item \textbf{Evaluation Bias}. When inappropriate or disproprtionate benchmarks are used in model evaluation, they can introduce the benchmarks biases into the model. \autocites{M144_Suresh_2021}{M24_Buolamwini_2018}
			
			\item  \textbf{Incorporation bias}. When index tests in diagnostic accuracy studies are part of the reference tests, this results in elevated sensitivity for the index tests \autocites{Chakraborty_2024, c21, c25, c26}{Young_2020}.
			
			\item \textbf{Popularity Bias}. More popular items tend to be exposed more. Popularity metrics can be manipulated though or not reflecting good quality, this can lead to bias \autocites{M117_Ciampaglia_2018}{Mehrabi_2021}.
			
			\item \textbf{Generalization Issues}.  \autocite{} \todo{add those from young}
		\end{itemize}
		
		
		\subparagraph{Potential Biases in PASSION}
		\todo{add}
		
		\rawcitationstart
		used
		\begin{itemize}		
			\rawcitationusedstart
			\item Evaluation Bias. Evaluation bias happens during model evaluation \autocite{M144_Suresh_2021}. This includes the use of inappropriate and disproportionate benchmarks for evaluation of applications such as Adience and IJB-A benchmarks. These benchmarks are used in the evaluation of facial recognition systems that were biased toward skin color and sex \autocite{M24_Buolamwini_2018}, and can serve as examples for this type of bias \autocite{M144_Suresh_2021}. \autocite{Mehrabi_2021}. -- important for this thesis
			
			\item  Incorporation bias: This is principally relevant for diagnostic accuracy studies when the index test forms a part of the reference test, resulting in elevated sensitivity e.g., if one wants to compare the grattage test vs. dermoscopy in psoriasis and does dermoscopy only from areas of grattage positivity, one would get a very high sensitivity for the grattage test because it was incorporated into the reference test, i.e., dermoscopy.25,26 21.\autocite{Chakraborty_2024}
			
			\item Popularity Bias. Items that are more popular tend to be exposed more. However, popularity metrics are subject to manipulation—for example, by fake reviews or social bots \autocite{M117_Ciampaglia_2018}. ... this presentation may not be a result of good quality; instead, it may be due to other biased factors. \autocite{Mehrabi_2021}.
			\rawcitationusedend
		\end{itemize}
		\rawcitationend
		
		
		\subsection{User Biases}
		
		\paragraph{Cognitive Biases}
		Biases which are related to human perception belong to the category of cognitive biases. They are affecting how data should be presented and interpreted \autocite{Mester_2017}
		
		Those biases can be introduced :
		\begin{itemize}
			\item \textbf{Confirmation Bias}. When people have pre-conceptions, they will only listen to the part of presented information which reinforce those "facts", regardless whether the facts are true or not \autocite{Mester_2017}. In health-care, this can be observed when patients report increases in diseases due to potentially nonfactual information they found on the internet \autocite{Chakraborty_2024, c15, c14}.
			\item \textbf{Belief Bias}. A stronger version of the confirmation bias: Someone who is affected by this bias is so sure about their own gut feelings that they will ignore results of a data research project \autocite{Mester_2017}.
			\item \textbf{Previous Opinion Bias}. When performing multiple tests, the knowledge about the outcome of the previous tests probably influences the results \autocite{Chakraborty_2024}
			\item \textbf{Cause-Effect Bias}. The famous senctence "correlation does not imply causation" can be used here - when correlation between two variables is misinterpreted as a cause-effect in the wrong direction, this bias applies  \autocite{Mester_2017}
			\item \textbf{Historical Bias}. Preexisting biases in the world can affect the data generation process \autocite{M144_Suresh_2021}. Even if they reflect the current reality, it is worth to consider whether those biases should affect the algorithms in question \autocite{Mehrabi_2021}.
			\item \textbf{Content Production Bias}. User generated contents can introduce biases by systematical differences in the production process, stucture and appearance, which might stem from the users background \autocite{M120_Olteanu_2019}.
		\end{itemize}
		
		\subparagraph{Potential Biases in PASSION}
		For PASSION, confirmation bias could lead to issues in the initial diagnosis and could therefore lead to biased data labeling. Same with the previous opinion bias. The later can be reduced when it is ensured, that the labeling experts are diagnosing the diseases independently of each other, so that they do not know the previous opinions.
		Cause-Effect bias is lesser an issue for PASSION, since the causes of the diseases are not analyzed. It could more be an inherit problem, that the algorithm learns wrong causes for diseases, such as appearing hair
		Historical bias can affect PASSIONs process in various ways.
		In PASSION context, Content Production Bias could have an impact on how the images are taken.
		
		
		\rawcitationstart
		used
		\begin{itemize}		
			\rawcitationusedstart
			\item  	
			\item Cognitive bias \autocites{Mester_2017} - statistical bias
			
			\item  Previous opinion bias: In performing a second diagnostic test, if the result of a previous test is known, it is likely to influence the result. An extension of this is the Greenwald’s law of lupus: the Sontheimer amendment – anything and everything that happens to a lupus erythematosus patient is correctly or incorrectly attributed to lupus.32 29. \autocite{Chakraborty_2024} - dermatology
			
			\item  Confirmation bias: This bias occurs when study participants have a preconceived notion of their disease that may not be based on facts. For example, we have observed that in North India many tinea patients report an increase in their disease due to taking meat, fish, and other so-called “hot foods”. They may also present information they have collected from the internet which reinforces their beliefs.15 14.\autocite{Chakraborty_2024} - dermatology
			
			\item Cause-effect bias \autocites{Mester_2022}{Mester_2017} - statistical bias
			
			
			\item Historical Bias. Historical bias is the already existing bias and socio-technical issues in the world and can seep into from the data generation process even given a perfect sampling and feature selection \autocite{M144_Suresh_2021}. ... search results were of course reflecting the reality, but whether or not the search algorithms should reflect this reality is an issue worth considering \autocite{Mehrabi_2021} - maybe relevant
			
			\item Content Production Bias. Content Production bias arises from structural, lexical, semantic, and syntactic differences in the contents generated by users \autocite{M120_Olteanu_2019}. \autocite{Mehrabi_2021} -- could the quality of the pictures been related to this as well?	
			\rawcitationusedend
		\end{itemize}
		\rawcitationend
		
		\paragraph{Behavioral Biases}
		
		Those biases can be introduced :
		\begin{itemize}
			\item \textbf{Behavioral Bias}. User behaviour can differ depending on the platforms, contexts, cultures, or datasets \autocite{M120_Olteanu_2019}.
			\item \textbf{Self-Selection Bias}. This subtype of selection bias occurs when study participants can select themselves. Less proactive people, people with less time or interest will be excluded or underrepresented \autocites{Mester_2022}{Mehrabi_2021}. \textbf{Non-Responder bias} is a subtype, where part of the population is not responding e.g. to fill out a survey or post-study responses queried by postal services \autocite{Chakraborty_2024}. \todo{maybe categorize this in the data biases or the healthy volunteer bias here}
			\item \textbf{Social Bias}. When the actions of others affect our judgment, it is called social bias. For example ratings in juries can be affected by this \autocite{M9_Baeza-Yates_2018}.
		\end{itemize}
		
		
		\subparagraph{Potential Biases in PASSION}
		For PASSION the behavioral biases can affect who is going to the dermatologists for what reasons. Therefore, the approach to use data from different countries may be benefitial, since potentially the cultural differences could differ.
		Self-selection is an issue, since only those patients can be included in the database which go to the hospitals.
		
		
		
		\rawcitationstart
		used
		\begin{itemize}		
			\rawcitationusedstart
			\item Self-Selection Bias. Self-selection bias4 is a subtype of the selection or sampling bias in which subjects of the research select themselves. \autocite{Mehrabi_2021}
			\item Self-selection bias - when you let the subjects of the analyses select themselves, less proactive people will be excluded \todo{could be an issue as well for PASSION, couldn't it? since the doctors probably ask the clients. One way to go is to default should be to provide access to the data. but is it ethical?} \autocites{Mester_2022}{Mester_2017}- statistical bias
			A variation of this is non-responder bias, where non-responders to a questionnaire differ significantly from responders.9 9. \autocite{Chakraborty_2024} - dermatology
			
			\item Social Bias. Social bias happens when others’ actions affect our judgment \autocite{M9_Baeza-Yates_2018}. (case where we want to rate or review an item with a low score, but when influenced by other high ratings, we change our scoring thinking that perhaps we are being too harsh [\autocite{M9_Baeza-Yates_2018}, \autocite{M151_Wang_2014}.) \autocite{Mehrabi_2021}
			
			\item Behavioral Bias. Behavioral bias arises from different user behavior across platforms, contexts, or different datasets \autocite{M120_Olteanu_2019}. \autocite{Mehrabi_2021} maybe, people from different countries go to the dermatologist for different diseases, based on cultural differences?
			\rawcitationusedend
			
		\end{itemize}
		\rawcitationend
		
		\paragraph{Publication Biases}
		
		Those biases can be introduced :
		\begin{itemize}
			\item \textbf{Publication Bias}
			\item \textbf{Hot stuff bias} is a subtype of publication bias, where Journals are less critical about trending topics, which lead to more frequent publishing of those topics. This in turn can lead to flawed meta-analyses regarding those topics  \autocite{Chakraborty_2024, c22, c23, c19}.
			\item \textbf{All is Well Bias}. This bias is a different view on the hot stuff bias. Theories which align with the view of the majority are more likely to be pubhlished than an opposing view \autocite{Chakraborty_2024, c7,c10-12}.
			\item \textbf{Rethoric Bias}. Charismatic writing or when the press is more vocal about findings can lead to greater influence over individuals than other available facts \autocite{Chakraborty_2024}.
			\item \textbf{Novelty Bias}. Newer interventions appear to be better. Over time, this effect decreases \autocite{Chakraborty_2024}.
		\end{itemize}
		
		\subparagraph{Potential Biases in PASSION}
		These biases are relevant for all researchers. They should kept in mind when interpreting, publishing and peer-reviewing papers.
		
		
		
		\rawcitationstart
		used
		\begin{itemize}		
			\rawcitationusedstart
			\item Hot stuff bias: Editors of journals may be less critical about topics that are “fashionable” or currently in vogue and consequently end up publishing them more frequently, resulting in publication bias as well as hot stuff bias. It can result in flawed meta-analyses based on these studies. An example is how cutaneous manifestations of COVID-19 were published. Indian Journal of Dermatology Venereology and Leprosy stood out by choosing not to publish anything and everything related to COVID-19, thus reducing hot stuff bias.22,23 19. \autocite{Chakraborty_2024}
			\item All is well bias: It is a subjective bias where theories supported by the majority tend to get more easily published than the opposing view supported by the minority. For example, ideas on the origin of endemic pemphigus supporting autoimmunity are more likely to be published than theories exploring an infectious trigger. According to some authors, this bias is very difficult to eliminate and is a variant of publication bias.10-12 7.\autocite{Chakraborty_2024} - dermatology
			
			\item Rhetoric bias: A more charismatic piece of writing has a greater influence on the study participants than other available literature. An example is the wider use of sunscreen for polymorphous light eruption over photoprotective strategies like umbrellas, broadbrimmed hats, etc, because the lay press is more vocal about sunscreens.14 11. \autocite{Chakraborty_2024} - dermatology
			
			\item  Novelty bias: The newer an intervention, the better it appears, and with time, its efficacy seems to decrease. When ligelizumab, an IgE antagonist was first discovered, ligelizumab was believed to be better than omalizumab; however, evidence soon pointed to the contrary. 16.\autocite{Chakraborty_2024} - dermatology			
			\rawcitationusedend
		\end{itemize}
		\rawcitationend
		
		
		\paragraph{Medical Biases}
		
		Those biases can be introduced :
		\begin{itemize}
			\item \textbf{Popularity Bias}. In medicine, when more popular diseases (usually well-known or stigmatized ones) get compared with less popular diseases, clinic rates can show a distorted view. The more popular diseases appear to be over-represented over more commoner ones \autocite{Chakraborty_2024, c9, c6}.
			\item \textbf{Apprehension Bias}. Fear related to an upcoming procedure can lead to false evaluations, e.g. when measuring blood pressure \autocite{Chakraborty_2024, c13}.
			\item \textbf{Hawthrone bias}. Subjects might modify their behaviour when they know they are being watched. This bias can be practically utilized by introducing regular follow-ups \autocite{Chakraborty_2024, c8}.
			\item \textbf{Centripetal Bias}. Better reputations affect to which physicians or hospitals patients tend to go to. Famous specialists probably see more cases in regards of their specialty than others \autocite{Chakraborty_2024, c12}.
		\end{itemize}
		
		
		\subparagraph{Potential Biases in PASSION}
		PASSION must be careful in interpreting the metadata. Since the data is from hospitals, they could be biased towards more popular diseases.
		PASSION can potentially use Hawthrone bias to improve the work of the annotators.
		Centripetal bias can also be used when selecting the partners to work with.
		
		
		\rawcitationstart
		used
		\begin{itemize}		
			\rawcitationusedstart
			\item Popularity Bias: This bias arises when a particular disease is more popular (i.e. either more well-known or more stigmatised) among the participants than the disease with which it is compared. For example, if a study compares clinic attendance rates among various dermatological disorders, one would see vitiligo patients are over-represented over melasma. While melasma is commoner in the normal population, vitiligo, due to its popularity because of media publicity and other factors, tends to present earlier.9 6. \autocite{Chakraborty_2024} - dermatology
			
			\item  Apprehension bias: This results from fear and apprehensions related to an impending procedure. The classic example is the false elevation of blood pressure because the person is apprehensive of his or her blood pressure being measured.13 A variant of this is the Hawthorne bias, where subjects modify their behavior, such as regularly taking a prescribed drug or exercising, simply because they know they are being watched, but not due to any apprehensions. Hawthorne bias is practically utilised in many leprosy clinics since regular follow-up has been shown to improve adherence to therapy based on Hawthorne bias. 8. \autocite{Chakraborty_2024} - dermatology
			
			\item Centripetal bias: Patients tend to go to more reputed physicians and hospitals than others. For example, a famous or better-known cosmetologist with a good reputation tends to see more cases than other cosmetologists. 12.\autocite{Chakraborty_2024}  - dermatology
			\rawcitationusedend
		\end{itemize}
		\rawcitationend
			
		
		
		\printbibliography[title=Appendix Bibliography]
	\end{refsection}
%\iftrue
%
\iffalse
\end{document}
\fi
