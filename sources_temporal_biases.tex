\rawcitationstart
used
\begin{itemize}		
	\rawcitationusedstart
	\item Longitudinal Data Fallacy. Researchers analyzing temporal data must use longitudinal analysis to track cohorts over time to learn their behavior. Instead, temporal data is often modeled using cross-sectional analysis, which combines diverse cohorts at a single time point. The heterogeneous cohorts can bias cross-sectional analysis, leading to different conclusions than longitudinal analysis \autocite{Mehrabi_2021}. --> could this be relevant for the progress of a specific disease? Or would that only be an issue when the progress of the disease would be predicted?
	\item  Chronological bias: This occurs in long-term studies where participants recruited earlier face different exposures and treatments than those recruited later. For example, biologics for psoriasis came later, so studies with long-term follow-up of psoriasis patients (say 30 years) will likely have this bias.9 13.\autocite{Chakraborty_2023}
	\item Immortal time bias: This type of bias is commonly encountered in cohort studies where the exposure has occurred, but the outcome cannot happen. Before the participant is assigned to the treatment group, the period between the exposure and the outcome occurrence is considered immortal time. For example, if a cohort study wants to look at relapse rates following successful treatment of psoriatic erythroderma while on methotrexate, one would follow them up from hospital discharge till the first readmission occurs. Once readmission occurs, they are assigned to treatment groups. This time gap when relapse has not occurred and the patient is on methotrexate constitutes immortal time bias.24 20.\autocite{Chakraborty_2023}
	\rawcitationusedend
\end{itemize}

even more extensive
\begin{itemize}
	\item 
\end{itemize}
