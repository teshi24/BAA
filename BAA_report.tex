\documentclass[a4paper,10.5pt,
			   bindingoffset=0.2in,left=3.35cm,right=2.12cm,top=3.75cm,bottom=2.88cm,%
				footskip=.25in
				listof=numbered,toc=chapterentrywithdots]{scrreport}

% Package Imports
\usepackage[utf8]{inputenc} % UTF-8 encoding
\usepackage[T1]{fontenc}    % Proper font encoding
\usepackage{lmodern}        % Modern font
\usepackage{graphicx}       % Graphics support
\usepackage[hidelinks]{hyperref}       % Clickable links
\usepackage{amsmath, amssymb} % Math symbols
\usepackage{listings}       % Code listings
\usepackage{csquotes}       % Quotation support
\usepackage{booktabs}       % Better tables
\usepackage{caption}        % Better captions
\usepackage{float}			% better positioning
\usepackage{tabularx}

%\documentclass[listof=numbered]{scrreprt} % making lists caputred

\usepackage[toc]{glossaries} % Glossary
\makeglossaries
\newglossaryentry{fitzpatrick-skin-type}{
	name={Fitzpatrick skin type},
	plural={Fitzpatrick skin types},
	description={A skin classifier based on the skins' reaction to ultraviolet light, developed by dermatologist Dr. Thomas Fitzpatrick \autocite{Gottfrois2024}}
}
\newglossaryentry{JupyterNotebook}{
	name={Jupyter Notebook},
	description={Executable files, often used in ML to write Python code and add explanations in text form}
}
\newglossaryentry{gpuhub}{
	name={GPUhub},
	description={\gls{HSLU}’s server infrastructure for GPU-related computing. It provides isolated environments with JupyterLab access for developing and running \gls{ML} workflows}
}
\newglossaryentry{pediatric}{
	name={pediatric},
	description={A medical term for infants, children and adolescents \autocite{Farlex_nodate}}
}
\newglossaryentry{proxyVar}{
	name={proxy variable},
	plural={proxy variables},
	description={"one or more variables that encode the protected attribute with a substantial degree of accuracy" \autocite{Wang_2021}}
}
\newglossaryentry{teledermatology}{
	name={teledermatology},
	description={dermatological care from a distance, supported by modern technology \autocite{Pala_2020}}
}
\newglossaryentry{Fairlearn}{
	name=Fairlearn,
	description={A Python library for assessing and improving fairness in machine learning models. It supports various fairness metrics and mitigation techniques, especially for binary classification tasks \autocite{Fairlearn_nodate}}
}
\newglossaryentry{Equalized-Odds-Difference}{
	name={equalized odds difference},
	description={The absolute difference in true positive and false positive rates between subgroups, used as a group fairness metric \autocite{Fairlearn_nodate}}
}
\newglossaryentry{Equalized-Odds-Ratio}{
	name={equalized odds ratio},
	description={The ratio of true positive and false positive rates between subgroups, used as a group fairness metric \autocite{Fairlearn_nodate}}
}
\renewcommand{\glossarysection}[2][]{}


\usepackage[style=apa,backend=biber]{biblatex}       % Bibliography
\addbibresource{references.bib} % Bibliography file

\usepackage{todonotes}
\usepackage{xcolor}
\renewcommand{\todo}[1]{\textcolor{red}{TODO: #1}}
%\usepackage{titlesec} for report
\renewcommand{\chapterformat}{\thechapter\autodot\enskip}


% Change chapter format in the document (no "Chapter", just number + title)
%\titleformat{\chapter}[hang]{\normalfont\huge\bfseries}{\thechapter}{1em}{}

% Ensure chapters still appear correctly in the Table of Contents
\renewcommand{\chaptermark}[1]{\markboth{\thechapter\ #1}{}}
\renewcommand{\thechapter}{\arabic{chapter}}
\renewcommand{\thesection}{\thechapter.\arabic{section}}


%\usepackage[titles]{tocloft}

% adapt toc
%\usepackage[nottoc]{tocbibind}
\usepackage{tocbasic} % Instead of tocbibind
%%\KOMAoptions{toc=chapterentrywithdots} % Ensures correct TOC formatting
%%\KOMAoptions{listof=leveldown} 

\setcounter{tocdepth}{2}


% Title Page
\title{Demographic Biases in Dermatology Models}
\author{Nadja Stadelmann}
\date{\today}

\begin{document}
	\pagenumbering{gobble}
	% Title Page
	\maketitle
	
	
	\fontsize{12}{14}
	\noindent\textbf{Bachelor Thesis at Lucerne University of Applied Sciences and Arts
		School of Computer Science and Information Technology} \\ \vspace*{0.6cm}
	
	\fontsize{10.5}{12}
	\noindent
	\textbf{Title of Bachelor Thesis:} Demographic Biases in Dermatology Models\\ \vspace*{0.2cm}
	
	\noindent
	\textbf{Name of Student:} Nadja Stadelmann\newline \newline
	\textbf{Degree Program:} BSc Computer Science \newline \newline
	\textbf{Year of Graduation:} 2025\newline \newline
	\textbf{Main Advisor:} Ludovic Amruthalingam\newline \newline
	\textbf{External Expert:} \newline \newline
	\textbf{Industry partner/provider:} Applied AI Research Lab, Lucerne University of Applied Sciences and Arts\newline \newline \newline
	\textbf{Code / Thesis Classification:}\\
	$\boxtimes$ Public (Standard) \newline
	$\square$ Private
	
	
	\todo{fix linebreaks and indents}
	\paragraph{\textbf{Declaration}}
	I hereby declare that I have completed this thesis alone and without any unauthorized or external help. I further declare that all the sources, references, literature and any other associated resources have been correctly and appropriately cited and referenced. The confidentiality of the project provider (industry partner) as well as the intellectual property rights of the Lucerne University of Applied Sciences and Arts have been fully and entirely respected in completion of this thesis. \newline \newline
	Place / Date, Signature	\underline{\hspace*{8cm}} \newline \newline
	
	\paragraph{\textbf{Submission of the Thesis to the Portfolio Database:}}
	
	Confirmation by the student \newline
	I hereby confirm that this bachelor thesis has been correctly uploaded to the Portfolio Database in line with the code of practice of the University. I rescind all responsibility and authorization after upload so that no changes or amendments to the document may be undertaken.  \newline \newline
	Place / Date, Signature	\underline{\hspace*{8cm}} \newline
	
	\newpage
	\paragraph{\textbf{Expression of thanks and gratitude}}
	
	\todo{add thanks and gratitude}
	
	Ludovic Amruthalingam
	
	Simone Lionetti - deputy Ludovic
	
	Pascal Baumann - LaTeX
	
	\newpage
	\pagenumbering{arabic}
	% Abstract
	\begin{abstract}
		\todo{Your abstract here.}
	\end{abstract}
	
	% Table of Contents
	\tableofcontents
	\listoftodos
	
	\todo{include the first three pages}
	\todo{Alle Fakten (fundiertes Wissen Dritter) sind korrekt zitiert. Es werden verschiedene Zitierweisen verwendet und teilweise mehrere Interpretationen gegenübergestellt. Der gemeinsam definierte Zitierstil im Text, in Abbildungen und Tabellen sowie im Literaturverzeichnis wird korrekt und durchgängig angewendet. Eigene Leistungen (sowie Bewertungen) und Fremdquellen sowie Recherchen sind klar unterscheidbar.} \linebreak
	\todo{Die erstellten Artefakte sind von sehr hoher Qualität. Das trifft u.a. auf Diagramme, Skizzen sowie Notationen (z.B. BPMN/UML) zu. Darstellungen sind einwandfrei, alle statistisch notwendigen Qualitätskriterien sind erfüllt. Beschriftungen etc. sind vorhanden, keine Einwände, Text und Bild stimmen beschreibend gut überein. Es wurden angemessene Dokumentationsmethoden und -arten korrekt verwendet. Vereinbarte Interview Transkripte, Beobachtungsprotokolle bzw. Zusammen-fassungen sind vorhanden. Daten, Ort, Kontext, Beschreibung, Zeilennummer, Verweise, Strukturen sind erkennbar, gut formatiert und korrekt mit dem Text/ der Analyse verknüpft. Alle Elemente und Themen sind im methodischen Teil/Text erklärt und verständlich, keine technischen oder strukturellen Einwände. Auch Zwischenanalysen, Zwischenschritte oder Gesamtauswertungen wurden durchgeführt, die Herkunft der Daten ist erkennbar und professionell aufbereitet.} \linebreak
	\todo{Der Schreibstil aller Dokumente entspricht hohen Standards und enthält keine Übertreibungen oder unbegründete Beurteilungen. Die Sprache ist aussagekräftig, prägnant und präzise. Die Fachterminologie ist konsistent, d.h. für gleiche Gegenstände und Themen werden immer die gleichen Begriffe verwendet. Der Sprachgebrauch ist durchgängig geschlechtergerecht, einheitlich und sachlich.}
	
	
	% Chapters
	\chapter{Problem Statement}
		\todo{Welche Ziele, Fragestellungen werden mit dem Projekt verfolgt? Die Bedeutung, Auswirkung und Relevanz	dieses Projektes für die unterschiedlichen Beteiligten soll aufgeführt werden. Typischerweise wird hier ein Verweis auf die Aufgabenstellung im Anhang gemacht.}
	
		\section{Context}
			This thesis is part of the PASSION project. The PASSION research team identified that in Africa, dermatology treatment is not accessible. There is less than one dermatologist per one million citizens. In contrast, there is high demand for dermatology treatment, especially among children and adolescents. 80\% of the \gls{pediatric} population is affected. The goal of PASSION is to make dermatology treatment more accessible by using AI supported telemedicine for triage \autocite{Gottfrois2024}.
		
			For AI supported triage, demographic biases in existing dermatology models is a problem since the corresponding datasets lack diversity, especially regarding skin tones \autocite{Gottfrois2024}. This type of bias is important in dermatology, since different diseases present themselves differently depending on the skin-color \autocite{Diaz2022}. Further, skin diseases are more advanced or severe at diagnosis in patients with lower socioeconomic status \autocite{BAD2021}.
			
			PASSION tries to mitigate the demographic bias by providing a dataset of pigmented skin images of patients from Sub-Saharan Africa. The PASSION team focused on gathering data with Fitzpatrick skin type (\gls{FST}) IV, V and VI. Further, the covered conditions represent up to 80\% of the conditions in the \gls{pediatric} population, the demographic group who is most affected by skin disease \autocite{Gottfrois2024}.
			
			The PASSION dataset is complementary to the existing datasets and improves the diversity in a combined dataset. Within the dataset itself, there could potentially be further demographic biases, e.g. related to age or gender.
			 
		\section{Objective}
		The goal of this research is to
		\begin{enumerate}
			\item Identify demographic biases in dermatology AI models, using established fairness metrics.
			\item Identify mitigation strategies to minimize these biases.
			\item Assess the effectiveness of the mitigation strategies.
		\end{enumerate}
		It is important to identify the existent biases first, so that the mitigation strategies can be \todo{proceed here to reason why you chose those objectives}
		
	
	\chapter{State of Research}
		\todo{Bezogen auf die eigenen Zielsetzungen und Fragestellungen soll aufgezeigt werden, wie andere dieses oder ähnliche Probleme gelöst haben. Worauf können Sie aufbauen, was müssen Sie neu angehen?	Wodurch unterscheidet sich Ihre Lösung von anderen Lösungen? Für wissenschaftlich orientierte Arbeiten sei hier explizit auf (Balzert, S. 66 ff) verwiesen.}
		\todo{Relevante, aktuelle und fundierte Fachliteratur wurde identifiziert, kritisch geprüft und verwendet. Die Begriffe der Fragestellung sind definiert und referenziert. Der gesamte Kontext ist verknüpft und eine Abgrenzung wurde vorgenommen. All dies ist in einer leicht verständlichen Struktur formuliert und überprüft.}
	
		\section{PASSION for Dermatology}
			The PASSION research team provides a dataset including three analysis scripts and an AI model. For this thesis, it is important to understand which labels the dataset provides, so that the applicable bias mitigation methodologies can be chosen.
			
			The provided analysis scripts show a first insight into the demographic distribution in the dataset, such as Fitzpatrick skin type and cases per country distribution. The results of those analyses reveal first biases.
			
			There are also dermatology specific analysis scripts in regards of body localization by condition or impetigo cases. Those results 
			
			\subsection{PASSION Dataset}
			The PASSION dataset contains data from patients from four African countries. It contains 4901 images of dermatology cases and the corresponding demographic and clinical information, see \autoref{tab:PASSION_labels}. There is one record per patient and one or more corresponding images. The images are linked with the record by filename, which contains the subject\_id of the row entry. Access to the dataset can be requested via \href{https://passionderm.github.io/}{https://passionderm.github.io/} \autocite{Gottfrois2024}.
			
			\begin{table}[H]
				\centering
				\begin{tabularx}{\textwidth}{>{\hsize=.27\hsize}X>{\hsize=.27\hsize\raggedright}X>{\hsize=.46\hsize}X}
					\toprule
					\textbf{Label}       & \textbf{Data Type} & \textbf{Description}       \\ \midrule
					subject\_id          & string & Participant's unique identifier        \\
					country              & string & Country of origin of the participant   \\
					age                  & integer & Age of the participant in years       \\
					sex                  & m/f/o & Gender of the participant               \\
					fitzpatrick          & integer & \gls{FST}                \\
					body\_loc            & string (list; null-able, semicolon-separated) & Specific affected body locations \\
					impetig              & 0/1  & Presence of impetigo (1=present), may occur alone or with other conditions, affects the treatment options for coexisting conditions        \\
					conditions\_PASSION  & Eczema, Scabies, Fungal, Others & Primary diagnosed skin condition \\
					\bottomrule
				\end{tabularx}
				\caption{PASSION dataset - labels and descriptions \autocite{Gottfrois2024}}
				\label{tab:PASSION_labels}
			\end{table}
			
			\subsection{PASSION Analysis Scripts}
			With the Dataset, the PASSION research team provides a \gls{JupyterNotebook} with code examples and analysis scripts. They are listed in \autoref{tab:PASSION_scripts} with a description and an indicator, how relevant the scripts are for this thesis.
			\begin{table}[H]
			\centering
			\begin{tabularx}{\textwidth}{>{\hsize=.25\hsize\raggedright}X>{\hsize=.41\hsize}X>{\hsize=.34\hsize}X}
					\toprule
					\textbf{Script Title}       & \textbf{Description} & \textbf{Relevance - Reasoning}       \\ \midrule
		  				Linking CSV Data with Image Files & 
					  	Creates mapping between the data records and images. It further counts the cases by country  &
					  	\textbf{High} - Basis for other analysis's, potentially provides dermatological info        \\
					\hline
						Extracting and Comparing Subject IDs &
						Checks the dataset complecity and accurracy in regards of linking records and images &
						\textbf{Low} - Checks loaded data for completeness, but is not providing more insight   \\
					\hline
						Regrouping Malawi and Tanzania to EAS &
						data aggregation due to dataset size and geographical proximity &
						\textbf{Low} - Might be relevant to understand the dataset and for interpreting the results of the following scripts correctly  \\
					\hline
						Conditions by Country &
						Relationship between clinical conditions and country &
						\textbf{Medium} - Currently unsure whether this information is relevant for this thesis \todo{research relevance between country vs. clinical conditions in regards of demographic bias} \\
					\hline
						Body Localizations by Conditions &
						Shows correlation between the condition and primarily affected body parts; does not use all affected body parts listed in the data \todo{check with Philippe why this was done} &
						\textbf{Low} - While the correlation can be interesting for other research, it is not relevant for demographic biases. \\
					\hline
						Impetigo Cases            &
						Counts total number of impetigo cases as well as proportion to all cases &
						\textbf{Medium} - Currently unsure whether this information is relevant for this thesis \todo{research relevance between impedigo and demographic bias} \\
					\hline
						Distribution of Fitzpatrick Skin Types &
						Counts and visualizes the skin type distribution  &
						\textbf{High} - \gls{FST} is a demographic information   \\
					\bottomrule
				\end{tabularx}
				\caption{PASSION dataset - existing analysis scripts \autocite{Gottfrois2024} \todo{decide on a table style}}
				\label{tab:PASSION_scripts}
			\end{table}	
			
			\subsection{PASSION Experiments}
			see https://github.com/Digital-Dermatology/PASSION-Evaluation
			
			
		\section{General ML biases}
			\subsection{ML biases}
			
			\subsection{ML fairness metrics}
			
			\subsection{ML mitigation methods}
			
	
	\chapter{Ideas and Concepts}
	\section{sub chapter}
		\todo{Hier geht es um die Fragestellung, wie Sie die formulierten Ziele der Arbeit erreichen wollen. Sie halten z.B. erste, grobe Ideen, skizzenhafte Lösungsansätze fest. Gibt es mehrere Wege, Ansätze um dieses Ziel zu erreichen, begründen Sie hier, warum Sie einen bestimmten Weg einschlagen. Beispiel für ein Softwareprojekt: Erste Gedanken über eine grobe Systemarchitektur. Ist z.B. eine Microservice-Architektur angebracht? Welche Alternativen bestehen, wo gibt es Problempunkte? Die Umsetzung, die Beurteilung der Machbarkeit und die detaillierte Beschreibung der umgesetzten Architektur sind dann Teil der Realisierung.}
	
	\chapter{Methods}
		\todo{Hier halten Sie fest und begründen, welches Vorgehensmodell Sie für Ihr Projekt wählen. Sie verweisen allenfalls auf die daraus entstandenen, konkreten Terminpläne mit Meilensteinen, welche z.B. unter Realisierung (Kapitel 5) oder im Anhang versorgt sind. Bei Projekten mit einer verlangten wissenschaftlichen Tiefe werden hier die geplanten Forschungsmethoden wie quantitative/qualitative Interviews, Befragungen, Beobachtungen, Feldexperiment etc. beschrieben und begründet. Warum ist in Ihrer Situation ein Interview besser als eine Umfrage? Wer soll interview werden?}
		\todo{Die gewählten Methoden sind nachvollziehbar und begründet. Eine methodische Übersicht (Methodisches BigPicture) wurde aufgezeigt und Abgrenzungen erläutert.}
	
	\chapter{Execution}
		\todo{Dies ist das Hauptkapitel Ihrer Arbeit! Hier wird die Umsetzung der eigenen Ideen und Konzepte (Kapitel 3) anhand der gewählten Methoden (Kapitel 4) beschrieben, inkl. der dabei aufgetretenen Schwierigkeiten und Einschränkungen.}
		\todo{Die gewählten Methoden werden systematisch, konsistent und korrekt auf den Kontext der Arbeit angewendet. Die Bearbeitungs- bzw. Forschungsobjekte sind einheitlich benannt, im Kontext dargestellt und sinnvoll in die Arbeit integriert. Praxis- und Erfahrungswissen (z.B. aus Interviews) wird zur Validierung und Ergänzung der erarbeiteten Ergebnisse herangezogen. }
	
	\chapter{Evaluation and Validation}
		\todo{Auswertung und Interpretation der Ergebnisse. Nachweis, dass die Ziele erreicht wurden, oder warum	welche nicht erreicht wurden.}
		\todo{Die Ziele / Forschungsfragen sind dem Umfang der Arbeit entsprechend sehr klar abgegrenzt; sie sind präzise, überprüfbar und nach den Standards der Zielformulierung definiert. Die Zielerreichung wurde systematisch und korrekt validiert.}
		\todo{Die Herleitung und Bedeutung der Ergebnisse, mögliche Varianten, Gütekriterien und eine Validierung allgemein werden nachvollziehbar diskutiert}
	
	\chapter{Outlook}
		\todo{Reflexion der eigenen Arbeit, ungelöste Probleme, weitere Ideen.}
		\todo{Die Ergebnisse und Empfehlungen schaffen einen konkreten Mehrwert für die Auftraggebenden. Einschränkungen und Grenzen werden kritisch diskutiert und die nächsten Schritte im Ausblick festgehalten, so dass die Ergebnisse direkt in der Praxis weiterverwendet und/oder angewendet werden können.}
	
	% Lists and References
	\chapter{\glossaryname}
	\printglossary[title={}]


	\todo{Add to ToC of content somehow and fix chapter numbers}
	

	\listoffigures
	
	\listoftables
	\todo{Add List of Formulas if necessary}
	\todo{add AI declarations somewhere}
	
	
	\chapter{\bibname}
	\printbibliography[heading=none]

	
	% Appendix
	\appendix
	\chapter{Appendix}
		\todo{Projektspezifisch können weitere Dokumentationsteile angefügt werden wie: Aufgabenstellung, Projektmanagement-Plan/Bericht, Testplan/Testbericht, Bedienungsanleitungen, Details zu Umfragen, detaillierte Anforderungslisten, Referenzen auf projektspezifische Daten in externen Entwicklungs- und Datenverwaltungstools etc.}
	
\end{document}
