\documentclass[a4paper,11pt]{article}
\usepackage[a4paper,margin=1in, bottom=1cm]{geometry}
\usepackage{graphicx}
\usepackage{hyperref}
\usepackage{array}
\usepackage{longtable}
\usepackage{fancyhdr}
\usepackage{amssymb}  % for checkmarks and crosses
\usepackage{xcolor}   % for coloring the symbols
\usepackage{pifont}   % for coloring the symbols
\usepackage{enumitem}
\setlist[itemize]{topsep=6pt, partopsep=0pt, itemsep=0pt, parsep=0pt, left=0pt}


\newcommand{\done}{\textcolor{green}{\ding{52}}}
\newcommand{\ongoing}{\textcolor{orange}{\ding{45}}}
\newcommand{\notstarted}{\textcolor{red}{\ding{56}}}


\raggedright

\pagestyle{empty}
\rhead{\thepage}
\renewcommand{\headrulewidth}{0pt}
%\fancyhf{}
%\lhead{Bachelor Thesis: Demographic Biases in Dermatology AI - Protocol Mid-Term Presentation}
%\chead{Nadja Stadelmann}
%\rhead{\today}
%\renewcommand{\headrulewidth}{0.4pt}

\begin{document}
	
	\begin{center}
		\Large \textbf{Protocol: Midterm Presentation Bachelor Thesis "Demographic Biases in Dermatology AI"}
	\end{center}
	
	\vspace{1em}
	
	\begin{tabular}{p{4cm}p{11.5cm}}
		\textbf{Date:} & 07 April 2025, 09:00–10:00 \\
		\textbf{Participants:} & Jürg Schelldorfer, Ludovic Amruthalingam, Nadja Stadelmann \\
		\textbf{Protocol written by:} & Nadja Stadelmann \\
	\end{tabular}
	
	\vspace{1em}
	
	\renewcommand{\arraystretch}{1.5}
	\begin{longtable}{|>{\raggedright\arraybackslash}p{3.5cm}|>{\vspace{-\baselineskip}}p{12cm}|}
		\hline
		\textbf{Topic} & \textbf{Details} \\
		\hline
		Clarifications regarding the PASSION Dataset and Model & 
		\begin{itemize}
			\item The PASSION dataset was collected by researchers.
			\item The PASSION model is a plain ResNet50 architecture which was trained on the PASSION dataset.
			\item The PASSION model predicts dermatological conditions stored in the labels \textit{conditions\_PASSION} 
			(eczema, scabies, fungal or others) and \textit{impetig} (presence of impetigo) based on the input picture.
			\item The PASSION model is not yet used in practice.
			\item Bias in the PASSION model should be reduced, so that the model can serve as a benchmark model to assess other dermatology models in regards of fairness; highlighting biases.
		\end{itemize} \\
		\hline
		
		General Advice &
		\begin{itemize}
			\item It is important to be precise e.g. regarding whether one talks about the dataset or the model -> clearly differentiate them.
			\item It is important to be knowledgeable when talking about biases and fairness, since it is a very diverse area.
			\item Take into consideration the technical, but also the dermatological aspects. E.g. under-representation of different ages in a dataset is only an issue, if the disease presentation differs in reality based on the age of a patient.
		\end{itemize} \\
		\hline
		
		Bias in Models vs. Representation in Datasets &
		\begin{itemize}
			\item The provided definition of bias in the context of AI is good - keep in mind that it is focusing on the model's output only.
			\item Even if the dataset is skewed in regards of representation, the models output can still be unbiased (= fairness metrics report fairness over subgroups). That's why a dataset itself is not "biased".
			\item On the other hand, an unbiased model does not necessarily mean that the dataset is fully inclusive and has no limitations.
			\item For the dataset, it's a question how representative the dataset is for given subgroups.
		\end{itemize} \\
		\hline
		
		Dataset Limitations &
		\begin{itemize}
			\item It's important to know (and clearly state) the limitations of the dataset (e.g. representation issues, what data is in-, what out-of-distribution)
			\item In practice, when provided data belongs to a out-of-distribution-case, no result is provided by the AI model. (This is especially important in health care, since now false-positive diagnoses should be provided.)
		\end{itemize} \\
		\hline
		
		Mitigation Methods &
		\begin{itemize}
			\item Oversampling is not good practice, as it can lead to misleading results. (Was intended to serve as an example only in the presentation)
		\end{itemize} \\
		\hline
		
		Report &
		\begin{itemize}
			\item The list of relevant biases and mitigation methods which will be provided to the PASSION team can be added to the appendix. In the main report, focus on top 5 to 10 items.
		\end{itemize} \\
		\hline
	\end{longtable}
	
\end{document}
